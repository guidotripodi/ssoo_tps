
\subsection{Ejercicios}
\begin{itemize}
 \item 
\textbf{Ejercicio 3}
En segundo lugar, deberán implementar el servidor de backend multithreaded inspirándose en el código provisto y lo 
desarrollado en el punto anterior.
\end{itemize}

\subsection{Resultados y Conclusiones}


\subsubsection[Resolución Ejercicio 3]{Ejercicio 3}

\indent En este Ejercicio, se solicit\'{o} la implementación de un backend mutithreaded, para el desarrollo del mismo,
utilizamos la base del backend mono para la conexi\'{o}n con el servidor, el parseo de las fichas, tanto para la validez de las mismas
y tambi\'{e}n el env\'{\i}o de dimensiones del tablero de juego.\\

Utilizamos la función $atendedor\_de\_jugador$ la cual la convertimos en un thread para cada jugador. Esta función es llamada desde
la función main cuando las conexiones del socket entre el cliente-servidor son correctas. \\
Convertimos la función enunciada de la siguiente manera:\\
\begin{verbatim}
 pthread_create(&threads[i], NULL, &atendedor_de_jugador, (void *)&td[i]);
\end{verbatim}

En donde, threads es un arreglo de thread$\_$t y se le asigna uno a cada jugador y td es un arreglo de una estructura creada por
nosotros la cual contiene:
\begin{itemize}
 \item socket$\_$cliente$\_$struct Este es el socket correspondiente a cada jugador
 \item rw$\_$lock Esto es un read-write-lock que tendra cada jugador
\end{itemize}

Con esta estructura, la cual es creada fuera del ciclo en el que se cargan todos los jugadores, y luego por cada iteracion 
del ciclo se va guardando el socket y rw$\_$lock correspondiente.\\
De esta forma, vamos creando cada thread con sus respectivos socket y rw\_lock.\\

A continuación, mostraremos esta sección de código:\\
\begin{verbatim}
\*creacion de arreglo de threads y arreglo de estructuras thread_data *\

                      pthread_t threads[NUM_THREADS];
                      struct thread_data td[NUM_THREADS];
                      
\* carga por cada iteracion de ciclo y creacion del thread *\

                      td[i].socket_cliente_struct = socketfd_cliente;
                      td[i].rw_lock = read_write_lock;
                      pthread_create(&threads[i], NULL, &atendedor_de_jugador, (void *)&td[i]);
                      i++;
\end{verbatim}

Luego, en la función $atendedor\_de\_jugador$ la cual recibe un thread$\_$data lo guardamos en un puntero a thread$\_$data llamado
my$\_$data y creamos un entero llamado socket$\_$fd el socket que nos viene como parametro.\\

Luego, basandonos en el backend mono, realizamos una implementación similar con la particularidad que, en el if donde se
consulta si el mensaje del jugador es una ficha, palabra o update.\\
En caso de ser una ficha, luego de parsear el casillero, antes de chequear la validez de la ficha utilizamos nuestro read\_write\_lock y realizamos
la funcion $rlock()$. En caso de ser una ficha valida, realizamos un read unlock y procedemos a escribir de la siguiente manera:\\
\begin{verbatim}
                     my_data->rw_lock.wlock();
                     palabra_actual.push_back(ficha);
                     tablero_letras[ficha.fila][ficha.columna] = ficha.letra;
                     my_data->rw_lock.wunlock(); 
\end{verbatim}
Para luego enviar la misma y terminar la jugada. En caso de que la validez de la ficha no sea correcta, dejamos de leer
y procedemos a escribir para quitar fichas y asi dejar de escribir, como mostramos a continuación:\\

\begin{verbatim}
                     my_data->rw_lock.runlock();
                     my_data->rw_lock.wlock();
                     quitar_letras(palabra_actual);
                     my_data->rw_lock.wunlock(); 
\end{verbatim}

Por consiguiente, en caso de que el mensaje sea una palabra se realiza un wlock para escribir la palabra y luego un wunlock.\\
Finalmente, en caso de ser update se realiza un rlock para actualizar la pantalla y se finaliza la jugada.\\

Fuera de este IF, se finaliza el thread creado con la funcion $pthread_exit(NULL)$.\\

De esta manera, queda implementado nuestro backend multithreaded como fue solicitado.






