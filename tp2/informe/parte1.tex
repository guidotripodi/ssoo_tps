
\subsection{Ejercicios}
\begin{itemize}
 \item \textbf{Ejercicio 1 }
 En primer lugar, deberán implementar un Read-Write Lock libre de inanición utilizando únicamente Variables de Condición POSIX 
 y  respetando la interfaz provista en los archivos backend-multi/RWLock.h y backend-multi/RWLock.cpp
\end{itemize}

\subsection{Resultados y Conclusiones}

\subsubsection[Resolución Ejercicio 1]{Ejercicio 1}

\indent Nuestra implementación del Read-Write Lock se baso en el pseudocodigo implementado en el libro $The$ $Little$ $Book$ $of$ $Semaphores$
al resolver la inanición producida en el problema de $Readers-writers$.\\

Se utilizaron 3 Semaphores, los cuales son:\\
\begin{itemize}
 \item roomEmpty
 \item turnstile
 \item readers$\_$mutex
\end{itemize}

Y ademas, un entero denominado $readers$\\

Comenzando por la implementación de los lectores, el pseudocodigo del libro mencionado es el siguiente:\\

\begin{verbatim}
            turnstile.wait()
            turnstile.signal()

            readSwitch.lock(roomEmpty)
                # critical section for readers
            readSwitch.unlock(roomEmpty)
\end{verbatim}

De aquí nuestro código implementado fue el siguiente:\\

\begin{center}
            \textbf{READERS LOCK}     
\end{center}

 
\begin{verbatim}
            pthread_mutex_lock(&turnstile);
            pthread_mutex_unlock(&turnstile);

            pthread_mutex_lock(&readers_mutex);
            readers++;
            pthread_mutex_unlock(&readers_mutex);
\end{verbatim}

Como se puede observar en el codigo del lock del read, el lector realiza un lock (wait) y unlock (signal) del Semaphores $turnstile$
para tener su turno y que ningún otro lo saque.\\
Por consiguiente, se realiza el lock del mutex que se encuentra vinculado al entero $readers$, ya que este será
aumentará su cantidad en 1 para que de esta manera nadie pueda modificarlo, y luego es liberado dicho mutex ($readers\_mutex$).\\

Luego, nuestro $READ$ $UNLOCK$ fue el siguiente:\\

\begin{center}
            \textbf{READERS UNLOCK}
\end{center}

 
\begin{verbatim}
            pthread_mutex_lock(&readers_mutex);
            readers--;
            if (readers == 0) {
                pthread_cond_signal(&room_empty);		
            }
            pthread_mutex_unlock(&readers_mutex);
\end{verbatim}

En esta implementación, primero se realiza un lock del Semaphore vinculado al entero $readers$ ya que este disminuirá en 1.\\
Luego, se realiza una consulta chequeando el valor del entero, si este es 0 se le dara un signal al Semaphore $room\_empty$
para notificarle al escritor que ya no queda ningun lector y puede proceder a escribir.\\
Por último, se libera el mutex $readers\_mutex$.\\

Continuando con el escritor, el pseudocodigo fue el siguiente:\\

\begin{verbatim}

            turnstile.wait()
            roomEmpty.wait()
                # critical section for writers
            turnstile.signal()

            roomEmpty.signal()

\end{verbatim}

De aquí, nuestra implementación final fue:\\

\begin{center}
            \textbf{WRITERS LOCK}
\end{center}

 
\begin{verbatim}
            pthread_mutex_lock(&turnstile);
            pthread_mutex_lock(&readers_mutex);
            while(readers != 0)
                  pthread_cond_wait(&room_empty, &readers_mutex);
            pthread_mutex_unlock(&readers_mutex);
\end{verbatim}

Inicialmente en nuestra implementación del WRITE LOCK, se realiza un lock del Semaphore $turnstile$ para que nadie pueda quitarle el turno, se realiza un
lock del mutex vinculado al entero, y luego se ingresa a un ciclo siempre que $readers$ sea distinto de 0, esto se realiza
para luego poder ejecutar la funcion $pthread\_cond\_wait$ para que esta misma tenga un funcionamiento correcto y seguro al
chequear la condición sobre $room\_empty$.\\
Una vez que se salga del ciclo o no se ingrese al mismo se libera el mutex $readers\_mutex$.\\

Luego, la implementación del unlock fue:\\
\begin{center}
           \textbf{WRITERS UNLOCK}
\end{center}

 
\begin{verbatim}
           pthread_mutex_unlock(&turnstile);
\end{verbatim}

Para la implementación del unlock del writer solo se libera el Semaphore $turnstile$.\\

De esta manera, con dicha implementación, siempre que llegue un escritor el mismo tendrá su turno sin producirse inanición.
Ya que, en caso de haber lectores y llegar un escritor, estos terminaran de leer y en caso 
de llegar nuevos lectores deberán esperar a que el escritor finalice su ejecución.\\
