
\subsection{Makefile}

Nuestro Makefile fue implementado para poder realizar la experimentaci\'on por ejercicio, eliminadolos individualmente y/o todos juntos.\\
Los nombres son:\\
\begin{itemize}
 \item ejercicio1: make ejercicio1 y make clean ejercicio1.
 \item ejercicio2: make ejercicio2 y make clean ejercicio2.
 \item ejercicio3: make ejercicio3 y make clean ejercicio3.
 \item ejercicio4: make ejercicio4 y make clean ejercicio4.
 \item ejercicio5: make ejercicio5 y make clean ejercicio5.
 \item ejercicio6: make ejercicio6 y make clean ejercicio6.
 \item ejercicio7: make ejercicio7 y make clean ejercicio7.
 \item ejercicio8: make ejercicio8 y make clean ejercicio8.
\end{itemize}

Al hacer make, se crear\'a una carpeta con el ejercicio y dentro de la misma se encontrara/n el/los gr\'afico/s
Para eliminar todos los gr\'aficos y directorios hacer make cleanTodosLosEjercicios, igualmente este make clean est\'a implementado
para eliminar carpetas y/o archivos si estos existiesen o no.\\

\subsection{Bibliografia}

\begin{itemize}
 \item C\'atedra de Sistemas Operativos - Clases te\'oricas y pr\'acticas (2º Cuatrimestre 2015)
 \end{itemize}